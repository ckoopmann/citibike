\documentclass{beamer}
\usepackage{graphicx}
\usepackage[latin1]{inputenc}
\usepackage[T1]{fontenc}
\usepackage[english]{babel}
\usepackage{listings}
\usepackage{xcolor}
\usepackage{eso-pic}
\usepackage{mathrsfs}
\usepackage{url}
\usepackage{amssymb}
\usepackage{amsmath}
\usepackage{multirow}
\usepackage{hyperref}
\usepackage{booktabs}
% \usepackage{bbm}
\usepackage{cooltooltips}
\usepackage{colordef}
\usepackage{beamerdefs}
\usepackage{lvblisting}
\usepackage{listings}

\usepackage{environ}
\NewEnviron{myresizeenv}{\resizebox{\linewidth}{!}{\BODY}}
\NewEnviron{heightresizeenv}{\resizebox*{!}{0.6\textheight}{\BODY}}

\pgfdeclareimage[height=2cm]{logobig}{logo}
\pgfdeclareimage[height=0.7cm]{logosmall}{logo}

\renewcommand{\titlescale}{1.0}
\renewcommand{\titlescale}{1.0}
\renewcommand{\leftcol}{0.6}

\title[Analysis of Citibike Data]{Analysis of Citibike Data}
\authora{Christian Koopmann}
\institute{\href{https://github.com/ckoopmann/citibike}{\underline{Github-Repository}}}
\authorb{}
\authorc{}


\def\linka{}
\def\linkb{}
\def\linkc{}

\hypersetup{pdfpagemode=FullScreen}

\AtBeginSection[]{
  \begin{frame}
  \vfill
  \centering
  \begin{beamercolorbox}[sep=8pt,center,shadow=true,rounded=true]{title}
    \usebeamerfont{title}\insertsectionhead\par%
  \end{beamercolorbox}
  \vfill
  \end{frame}
}

\begin{document}
\lstset{language=R}
% 0-1
%%%%%%%%%%%%%%%%%%%%%%%%%%%%%%%%%%%%%%%%
\frame[plain]{

\titlepage
}

% No Number on the Outline Slide
\useheadtemplate{%
    \raisebox{-0.75cm}{\parbox{\textwidth}{%
            \footnotesize{\color{isegray}%
                \insertsection\ \leavevmode\leaders\hrule height3.2pt depth-2.8pt\hfill\kern0pt\ }}}
}



% 0-3
%%%%%%%%%%%%%%%%%%%%%%%%%%%%%%%%%%%%%%%%
\frame{
\frametitle{Outline}

\tableofcontents{}

}

\section{Exploratory Analysis}

\frame{
\frametitle{Overview}
\begin{itemize}
 \item Ca. 17.5 mio records of citibike trips in 2018
 \begin{itemize}
     \item  Network of 819 stations
     \item  Fleet of 15244 bikes
 \end{itemize}
 \item Downloaded as monthly csv-files then aggregated in SQL-Lite database
 \item 14 Variables from 3 categories (excl. bike-id):
 \begin{itemize}
     \item \underline{Demographics:} birth year, gender, usertype
     \item \underline{Time:} start/stop-time, trip duration
     \item \underline{Location:} id, name, latitude / longitude of start- and end-station
 \end{itemize}
\end{itemize}
}

\frame{
\frametitle{Customers seem to be older, more diverse and take their time}
% latex table generated in R 3.5.2 by xtable 1.8-4 package
% Thu Feb 13 06:44:41 2020
\begin{table}[ht]
\begin{myresizeenv}
\begin{tabular}{lrrrrrr}
  \toprule
Type & Trips[mio] & Age[y] & Duration[min] & Unknown[\%] & Male[\%] & Female[\%] \\ 
  \midrule
Subscriber & 15.60 & 38.59 & 13.20 & 1.95 & 75.10 & 24.90 \\ 
  Customer & 1.90 & 42.39 & 42.98 & 61.00 & 62.89 & 37.11 \\ 
   \midrule
All & 17.50 & 39.01 & 16.48 & 8.46 & 74.52 & 25.48 \\ 
   \bottomrule
\end{tabular}
\end{myresizeenv}
\end{table}

}

\frame{
    \frametitle{60 Percent of all customers born in 1969 ?}
    \begin{figure}
        \includegraphics[width=0.8\textwidth]{\string"../plots/histogram_age_by_type\string".jpeg}
    \end{figure}
}


\frame{
    \frametitle{Users with unknown gender seem to have a data quality problem}
    \begin{figure}
        \includegraphics[width=0.8\textwidth]{\string"../plots/histogram_age_by_type_gender\string".jpeg}
    \end{figure}
}

\frame{
\frametitle{Excluding data quality issues customers are actually younger}
% latex table generated in R 3.5.2 by xtable 1.8-4 package
% Thu Feb 13 10:24:20 2020
\begin{table}[ht]
\begin{myresizeenv}
\begin{tabular}{lrrrrr}
  \toprule
Type & Trips[mio] & Age[y] & Duration[min] & Male[\%] & Female[\%] \\ 
  \midrule
Subscriber & 15.30 & 38.45 & 13.16 & 75.10 & 24.90 \\ 
  Customer & 0.80 & 32.25 & 47.28 & 62.89 & 37.11 \\ 
   \midrule
All & 16.10 & 38.15 & 14.76 & 74.52 & 25.48 \\ 
   \bottomrule
\end{tabular}
\end{myresizeenv}
\end{table}

}


\frame{
\frametitle{Wider distribution of travel times for customers}
\begin{figure}
\includegraphics[width=0.8\textwidth]{\string"../plots/density_durations_by_type\string".jpeg}
\end{figure}
}

\frame{
\frametitle{Commuting subscribers vs. vacationing customers}
\begin{figure}
\includegraphics[width=0.8\textwidth]{\string"../plots/histogram_hour_by_type\string".jpeg}
\end{figure}
}

\frame{
\frametitle{Commuting subscribers vs. vacationing Customers}
\begin{figure}
\includegraphics[width=0.8\textwidth]{\string"../plots/histogram_weekday_by_type\string".jpeg}
\end{figure}
}

\frame{
\frametitle{Most popular starting stations for customers in tourist hotspots}
\begin{figure}
\includegraphics[height=0.8\textheight]{\string"../plots/start_station_map\string".jpeg}
\end{figure}
}

\frame{
\frametitle{Conclusions}
\begin{columns}
    \begin{column}{0.48\textwidth}
        \begin{center}
            \underline{Customer}
        \end{center}
    \end{column}
    \begin{column}{0.48\textwidth}
        \begin{center}
            \underline{Subscriber}
        \end{center}
    \end{column}
\end{columns}
\begin{columns}
    \begin{column}{0.48\textwidth}
        \begin{itemize}
            \item Longer trips
            \item Peak during weekend and afternoon hours
            \item High concentration of trips around tourist hotspots (Central Park)
            \item Data Quality Issues regarding demographic data
        \end{itemize}
    \end{column}
    \begin{column}{0.48\textwidth}
        \begin{itemize}
            \item Shorter trips 
            \item Peak during the week and morning/evening rush hours
            \item Distributed across variety of stations in the business district
        \end{itemize}
    \end{column}
\end{columns}
}

\section{Classification Model}

\frame{
\frametitle{Overview}
\begin{itemize}
    \item Binary target variable (Subscriber=1, Customer=0)
    \item Nine input variables chosen / constructed based on exploratory analysis (see next)
    \item Tested modelling algorithms:  Logistic Regression and Random Forest
    \item Randomly selected 100k/10k training/test observations
    \item Each algorithm tested on full feature set as well as reduced set without demographic variables
\end{itemize}
}

\frame{
\frametitle{Based on exploratory analysis nine features are included in the model training}
\begin{itemize}
    \item \textit{usertype}: Binary target variable (Subscriber=1, Customer=0)
    \item \textit{tripduration}: Integer variable from the original dataset in seconds
    \item \textit{age}: Integer variable based on birth year (demographic)
    \item \textit{gender-known, female}: Binary demographic variables based on original gender variable
    \item \textit{top-customer-station, top-subscriber-station}: Binary variable indicating wether trip started in one of the top 50 stations for each type
    \item \textit{weekend, morning-rushour, evening-rushour}: Binary variable based on starttime
\end{itemize}
}

\frame{
    \frametitle{Exluding demographic variables decreases performance significantly}
    \begin{figure} \includegraphics[width=0.8\textwidth]{\string"../plots/roc_plot\string".jpeg} \end{figure} 
}

\frame{
\frametitle{Logistic regression coefficients confirm intuition / exploratory results}
% latex table generated in R 3.5.2 by xtable 1.8-4 package
% Mon Feb 24 13:10:28 2020
\begin{table}[ht]
\begin{heightresizeenv}
\begin{tabular}{rrr}
  \toprule
 & Estimate & Pr($>$$|$z$|$) \\ 
  \midrule
(Intercept) & -2.6716 & 0.0000 \\ 
  tripduration & -0.0002 & 0.0000 \\ 
  age & 0.0369 & 0.0000 \\ 
  gender\_knownTRUE & 4.8340 & 0.0000 \\ 
  femaleTRUE & -0.5179 & 0.0000 \\ 
  weekendTRUE & -0.6826 & 0.0000 \\ 
  morning\_rushhourTRUE & 1.3339 & 0.0000 \\ 
  evening\_rushhourTRUE & 0.3927 & 0.0000 \\ 
  top\_subscriber\_stationTRUE & 0.5572 & 0.0000 \\ 
  top\_customer\_stationTRUE & -1.0828 & 0.0000 \\ 
   \bottomrule
\end{tabular}
\end{heightresizeenv}
\end{table}

}


\section{Travel Times}
\frame{
\frametitle{Overview}
\begin{itemize}
    \item Compare duration of citybike trips to reference trip times predicted by \textit{Google Maps}
    \item Evaluate 1000 randomly selected citybike trips that did not start and end at the same station
    \item \textit{Google Maps} reference times: 
        \begin{itemize}
            \item Route from start to end station coordinates
            \item Starttime: next timepoint in the future with same weekday, hour and minute
            \item Shortest route for travelmode \textit{transit} and \textit{driving}
        \end{itemize}
\end{itemize}
}


\frame{
    \frametitle{Sighteseeing customers are much slower than commuting subscribers}
    \begin{figure}
        \includegraphics[width=0.8\textwidth]{\string"../plots/plt_mean_diff_period\string".jpeg}
    \end{figure}
}

\section{Motor Vehicle Collision Data}

\frame{
\frametitle{Overview}
\begin{itemize}
    \item Data on 1.65 mio. crashes involving motor vehicles
    \item Features include:  Injured / Killed cyclists, pedestrians motorists as well as time and location of incident.
        \begin{itemize}
            \item Type of Involved vehicles
            \item Causal factors (alcohol, speeding etc.)
            \item Injured / Killed cyclicsts, motorists pedestrians
            \item Time and location of the accident
        \end{itemize}
    \item Potential insurance use case might be offering single trip insurance based on estimated accident risk.
\end{itemize}
}

\frame{
\frametitle{Most dangerous street crossings could be used as a risk indicator}
\begin{figure}
\includegraphics[height=0.8\textheight]{\string"../plots/crash_map\string".jpeg}
\end{figure}
}

% No number on outline slide
\useheadtemplate{%
    \raisebox{-0.75cm}{\parbox{\textwidth}{%
            \footnotesize{\color{isegray}%
                \insertsection\ \leavevmode\leaders\hrule height3.2pt depth-2.8pt\hfill\kern0pt\ \thesection-\thepage}}}}
\setcounter{section}{1}


\end{document}
